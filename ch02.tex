\topglue 0pt
\vskip 2cm
\centerline{CHAPTER II}
\vskip 2cm
\centerline{\bf RADIOACTIVITY AND NAURAL}
\centerline{\bf RADIOACTIVE ELEMENTS}
\vskip 2cm
{\sc For understanding the} theoretical and technical basis of radiometric methods, it is necessary to have a basic knowledge of radioactivity, the process and nature of radioactive disintegration etc.
Similarly, knowledge of the radioactive properties of minerals and rocks, and of the mode of occurrence of radioactive ore deposits of immense value for the proper choice of instruments and field procedures, and for the interpretation of the obtained data.
\bigskip
\noindent
{\sc 2.1 Atoms, Nuclei and Isotopes:}
\smallskip
An atom, which is the smallest fundamental unit of an element, consists essentially of two principal parts.
At the center of the atom is situated, a dense nucleus of very small dimensions (radius of the order of 10$^{-13}$ to 10$^{-12}$ cm).
Its mass is very nearly equal to that of the atom.
Around the nucleus are the electrons rotating in certain discrete orbits, much in the same way as the planets do around the sun.
The electrons are fine particles with probable dimensions similar to those of the smaller nuclei.
The mass of an electron (9.7$\times$10$^{-28}$ gm) is negligible compared to that of the nucleus.
The distance between the electrons and the nucleus is very large, in comparison to their dimensions so that a major part of the atomic space may be considered `empty'.
The overall radius is of the order of 10$^{-8}$ cm.
\smallskip
The electrons are negatively charged, while the nucleus is positively charged and the electric forces active between them (called coulomb forces) bind the atom together.
The charge of a single electron is 4.8$\times$10$^{-10}$ e.s.u. or 1.6$\times$10$^{-19}$ coulomb.
It is the smallest electric charge every observed, and for this reason is called an elementary charge ans as such has been assigned a unit value.
\smallskip
The nucleus essentially consists of two fundamental particles---the protons and the neutrons, collectively called the nucleons.
A proton has a single positive charge, numerically equal to that of an electron.
Normally, in an atom, there are as many protons as the electrons, so that the atom as a whole is electrically neutral.
The mass of a proton (1.67$\times$10$^{-24}$ gm) is equivalent to that of a hydrogen nucleus (which consists of only one proton) and has been assigned an arbitrary value of one mass unit.
\smallskip
The total number of protons in the nucleus of an atom is called the atomic number, designated $Z$.
The atomic number is also called the charge number, since it determines the positive charge of the nucleus.
The atomic number indirectly specifies the number of extranuclear electrons.
Since these orbital electrons are the active participants in the chemical processes between different elements (such as formation of compounds), the atomic number indirectly determines the chemical property of the element.
Each element is characterized by its atomic number whose value is fixed for the given element; e.g., for hydrogen $Z=1$, for helium $Z=2$, for uranium $Z=92$ etc.
\smallskip
The neutron is electrically neutral and has a mass slightly greater than that of the proton.
The total number of neutrons and protons is the nucleus is called the mass number, designated $A$.
Since the mass of an atom is nearly the same as that of its nucleus, the mass number represents the atomic weight.
Evidently, $A = Z + N$, where $N$ (called the neutron number) denotes the number of neutrons in the nucleus.
Thus, the mass number of hydrogen is 1; that of one type of uranium consisting of 92 protons and 143 neutrons is 235 and of another type of uranium with 92 protons and 146 neutrons is 238 and so on.
\smallskip
In nuclear transformations, such as radioactive disintegration, artificial transmutation of elements etc., we are mainly concerned with the nuclei of the elements rather than with the atoms.
In nuclear terminology, an atomic species, characterized by the constitution of its nuclei, i.e., by the number of protons and neutrons it contains, is sometimes referred to as a nuclide.
A nuclide exhibiting radioactivity is called a radionuclide.
\smallskip
It may so happen that the number of neutrons contained in the nuclei of atoms of an element may sometimes vary, as for instance in the case of two different types of uranium mentioned above.
Different types of an element having the same number of protons, but different neutron numbers are called isotopes.
Thus, atoms of different isotopes of an element possess the same atomic number, but different mass numbers.
Since the chemical properties of elements are essentially determined by their atomic numbers, the isotopes of an element are chemically similar and differ only in their mass numbers.
\smallskip
Almost all elements have stable isotopes; the number of isotopes of different elements ranges from 2 to 10 (the latter in the case of tin).
Usually, the isotopic abundance of elements, i.e., the amounts of isotopes present in an element is constant (except for radioactive isotopes).
The weighted average of the masses of all isotopes in an element is therefore its atomic weight.
\smallskip
Elements having the same mass number, but different atomic numbers are called isobars.
The atoms of isobars exhibit different chemical properties.
Elements such as argon ($Z=18$), potassium ($Z=19$) and calcium ($Z=20$) all of which have $A=40$, are examples of isobars.
\smallskip
Specification of the mass and atomic numbers of the atomic species defines the nature of the isotope of the element.
In nuclear nomenclature, it is customary to write the value of the atomic number $Z$ as a subscript and the mass number $A$ as a superscript of the symbol of the element $M$, such as $_ZM^A$.
Thus, we have the usual hydrogen $_1H^1$ and its isotopes $_1H^2$ and $_1H^3$ called deuterium and tritium respectively; the uranium isotopes $_{92}U^{238}$, $_{92}U^{235}$ and $_{92}U^{234}$, etc.
\bigskip
\noindent
{\sc Radioactivity:}
\smallskip
Every element has a preferred ratio of neutrons to protons, and if an isotope contains such number of protons and neutrons that their ratio is far different from the preferred ratio, the nucleus becomes unstable.
It is probable that such an unstable element transmutates spontaneously into another element in an attempt to form a stable configuration, releasing energy in the process, in the form of powerful corpuscular radiations of alpha or beta particles accompanied by electromagnetic (gamma) radiations.
The process of such transmutation of nuclei is called radioactivity.
If the probability of transmutation is non-existent or low, the element is either stable or weakly radioactive and if high, it is strongly radioactive.
Radioactivity is thus a spontaneous and self-destructive nuclear activity leading to the break-up of the element itself for good, i.e., an irreversible self-disintegration.
The activity is spontaneous in the sense it is intrinsic and if unaffected by external agents, either physical or chemical.
Further, the activity is not instantaneous, but is prolonged over a certain period of time, characteristic of the element.
\smallskip
Radioactive disintegration of an element results in the formation of a new element.
The new element this formed, if radioactive, disintegrates again to form another element.
In some cases, the process may go on covering a few elements until a stable one is formed.
Such radioactive elements are said to form radioactive series or family.
\bigskip
\noindent
{\sc 2.3. Rate of Radioactive Decay:}
\smallskip
The process of natural disintegration of radioactive elements is ascribed, as we have mentioned above, to the intrinsic instablility of their atomic nuclei.
This results in a violent break-up of nuclei from time to time and the ejection of alpha or beta particles.
\smallskip
The break-up of nuclei is purely a random process and it is impossible to predict when a particular nucleus disintegrates.
However, through a series of experiments, it was established that the rate of radioactive disintegration ($dN_t/dt$) of atoms of a given element at any instant of time ($t$) is proportional to the number ($N_t$) of atoms present at that instant, i.e.,
$$-{dN_t \over dt} = \lambda N_t \eqno 2.1$$
where $\lambda$ is the disintegration or decay constant, which represents the fraction of atoms that disintegrate in a unit time.
Its value is fixed for any given element and is not affected by any known physical or chemical process.
$\lambda$ has the dimensions of inverse time ($1/t$) and is expressed as sec$^{-1}$, min$^{-1}$, day$^{-1}$, year$^{-1}$ etc.
\smallskip
If, in a radioactive element, $N_0$ represents the number of atoms at a time $t=0$, the number $N_t$ of atoms remaining after a time $t$, can be obtained by integrating eq. 2.1, as
$$\int_{N_0}^{N_t} {dN_t \over N_t} = -\int_0^t \lambda dt$$
or
$$N_t = N_0 e^{-\lambda t} \eqno 2.2$$
This exponential relation governing the radioactive disintegrations is referred to as the law of radioactive decay.
The relation shows that the number of atoms which has escaped disintegration decreases with time at first rapidly and then more and more slowly, as shown in Fig. 1.
Theoretically speaking, it takes infinite time for any given radioactive element to disintegrate and disappear completely.
\smallskip
It may be noted that since the radioactive decay is a random process and is subject to fluctuations, the above experimental law is only a nearest approximation to reality.
It can be shown that the probability for the law to be strictly valid increases, as the total number of atoms involved becomes larger and larger.
\smallskip
The nuclei of different radioactive elements possess different degrees of instability and hence different radioactive elements have different decay constants. In addition to $\lambda$, two other constants, viz., the half-life period ($T$) and mean-life span of atoms ($\tau$) are used to describe the rate of radioactive decay.
\smallskip
The half-life period (also called for brevity as half-period or half-life) is defined as the period of time in which the number of atoms of a particular radioelement decreases to half its initial value (see Fig. 1).
The value of $T$ for any radioelement is a characteristic constant independent of all physical and chemical conditions.
The relation between $T$ and $\lambda$ can be obtained by replacing $T$ for $t$ in eq. 2.2, as
$$N_T = N_0 e^{-\lambda T} = {1 \over 2} N_0$$
from which we write
$$T = {\ln 2 \over \lambda} = {0.693 \over \lambda} \eqno 2.3$$
Thus, the half-life period is inversely proportional to the decay constant.
\smallskip
The law of radioactive decay does not give any indication of the disintegration pattern of individual atoms, since it is essentially a statistical law.
Among the disintegrating atoms, some may have only a short life while others may take a long time before they decay.
Since theoretically, it takes infinite time for any radioelement to disintegrate completely, we conclude that the life-span of an atom of radioelement can have any value between 0 and $\infty$.
In order to differentiate between various radioelements, the concept of mean-life span of atoms is introduced.
Mean-life span is defined as the ratio of the total life-span of all atoms and the number of atoms.
In practice, mean-life span represents average life-span of a large number of atoms.
\smallskip
Suppose there are $N_0$ atoms initially (at $t=0$).
After a time $t$, the number $dN$ of atoms that decay during a short period of time $dt$ (i.e., during the interval between $t$ and $t + dt$) is (from eqs. 2.1 and 2.2) $\lambda N_t dt$ or $\lambda N_0 e^{-\lambda t} dt$.
Each of these $dN$ atoms has a life $t$, so that the mean-life span of the whole number $N_0$ of atoms is given by
$$\eqalignno{
\tau = {1 \over N_0} \int_0^\infty t dN &= {1 \over N_0} \int_0^\infty t \lambda N_0 e^{-\lambda t} dt =&\cr
&\int_0^\infty t \lambda e^{-\lambda t} dt = {1 \over \lambda}& 2.4
}$$
Thus, the mean-life $\tau$ is the reciprocal of the decay constant $\lambda$.
Using eqs. 2.4 and 2.3, we can express $\tau$ in terms of $T$:
$$\tau = {T \over \ln 2} = {T \over 0.693} = 1.443 T \eqno 2.5$$
\bigskip
\noindent
{\sc 2.4. Successive Disintegration and Radioactive Equilibrium:}
\smallskip

\vfill 
\eject
