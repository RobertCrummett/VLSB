\topglue 0pt
\vskip 2cm
\centerline{CHAPTER I}
\vskip 2cm
\centerline{\bf INTRODUCTION}
\vskip 2cm
{\sc The discovery of} radioactivity by Henri Becquerel during the close of the nineteenth century and the subsequent researches, notably by Rutherford and Curies, marked the beginning of a new era of scientific progress.
The study of radioactivity has greatly enriched and enhanced our knowledge of the structure and constitution of the atom and the nucleus.
With it was born, the science of nuclear physics which through the twentieth century grew into gigantic proportions.
\smallskip
It is not only in the realm of theoretical knowledge that great progress has been made.
The application of radioactive techniques and the use of natural and artificial radioactive  substances in several diverse branches of science (like physics, chemistry, biology, medicine, astronomy, archaeology, meteorology, etc.) have proved significantly beneficial.
The technical and industrial uses of radioactivity are impressively many and it looks as though the possibilities are almost unlimited.
Some of these, like the production of electric power in atomic reactors and the treatment of malignant diseases by artificial radioactive sources, are well known.
\smallskip
Of concern to us is the application of radioactivity in geology and geophysics.
Firstly, since the radioactive elements present in the earth liberate energy in the form of heat during their decay, they contribute to the internal heat of the earth.
Study of the liberated energy and of the nature and amount of radioactive elements contained, leads to the appreciation of the thermal history of the earth.
Another important application of radioactivity is the determination of the age of the rocks and the earth.
By finding out the amounts of the decaying radioactive elements and the decay products in igneous rocks, their ages can be estimated since the rate of decay can be precisely measured.
Fairly accurate ages of several rocks have been obtained by utilizing the decay of uranium and thorium to lead, the decay of rubidium into strontium and the decay of potassium into argon.
\smallskip
An important application of radioactivity in geophysics is in the realm of exploration.
Methods involving the measurement of radioactivity of rocks and ores for prospecting purposes are called radiometric methods.
Among the three types of radiations (alpha, beta and gamma) emitted by radioactive elements, gamma rays are the most penetrating and are the only radiations that can reach the prospecting instruments.
Due to this reason, field radiometric methods of exploration mainly aim at the measurement of gamma intensity in an area.
These methods are called gamma radiometric methods or gamma methods.
The other radiometric methods that are widely used (usually in conjunction with gamma methods) are the emanation methods in which the alpha activity of radioactive emanations is measured.
\smallskip
The radiometric methods are mainly employed in the exploration of uranium and thorium mineral deposits.
Since both uranium and thorium are highly radioactive, the methods of their location involve direct search for areas of high radioactive intensity.
Due to the importance of uranium as a source material for nuclear energy, most of the prospecting work so far done by radiometric methods was aimed at the location of uranium mineral deposits.
In addition to uranium and thorium, potassium mineral deposits (which are also radioactive, though to a much lesser degree than uranium and thorium) are located by radiometric methods.
\smallskip
In recent times, the application of radiometric methods of exploration is more diversified.
They are used for the indirect location of deposits of certain rare earth elements (like zirconium, beryllium, niobium, yttrium etc.), and of certain rare earth metals occurring in genetic or paragenetic association with radioactive minerals.
Similarly, since different rocks contain varying amounts of radioactive minerals, the radiometric methods find application also in geological mapping, like delineation of contacts between different rock types, location of faults, shear zones etc., overlain by a thin soil cover.
More recently, some success has been reported in the utilization of radiometric field investigations for the direct location of oil and gas deposits.
\smallskip
The radiometric methods of exploration are now-a-days employed not only on the ground (foot and automobile surveys), but also from the air (airborne surveys) and in underground mines (gamma testing) and boreholes (gamma logging).
\smallskip
In the laboratory, alpha, beta and gamma radiometric methods are extensively used for assaying of ores and rocks for radioactive elements.
Such assaying techniques are non-destructive and rapid and are characterized by a high degree of accuracy, frequently comparable with that attainable in chemical and radiochemical methods of analysis.
Due to these reasons, they are often employed as the main methods for the industrial evaluation of radioactive deposits.
\smallskip
In all the above radiometric investigations, the intensity of radiation due to naturally occurring radioactive elements in rocks and ores is studied; but more recently, have been developed to utilize the artificially induced radioactivity in certain elements in rocks and ores in order to detect their presence during prospecting, mining operations and borehole logging, as also to estimate their concentration both in the laboratory and {\it in situ\/}.
These nuclear geophysical method, as they have been called, have been developed in some countries to such an extent that they are at present used as some of the routine tools in the investigation of a large number of minerals.
In this book, however, we are mainly concerned with the methods and techniques of radiometric exploration.
\vfill
\eject
